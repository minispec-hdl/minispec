\section{Structure of a Minispec Design}
\label{sec:design}

A Minispec design consists of one or more \emph{Minispec source files}.
Each source file should be a plain text file with extension \verb|.ms|.
Each source file can define its own used-defined types (\autoref{sec:userdefined}),
functions (\autoref{sec:functions}), modules (\autoref{sec:modules}),
and two other top-level statements not yet described: \emph{constants} and \emph{imports}.

\paragraph{Constants} are declared just like variables and must be initialized at declaration.
They may be read elsewhere but may not be changed. They are not state
(i.e., not like a global variable in other languages) and are simply a convenience
to avoid hardcoding constant values throughout the design.

\paragraph{Import} statements allow including the definitions from other files in the current one.
Their syntax is \verb|import file;|, where \verb|file| is the name of the imported file without the \verb|.ms| extension.
For example, \verb|import Example;| would import file \verb|Example.ms|.
\verb|file| can be uppercase or lowercase.
\verb|file| cannot include a directory: the imported file must be
in either the current directory or the compiler's search path (\autoref{sec:tools}).

Import statements may appear anywhere (they need not be at the top of the file).
An import statement is equivalent to including the contents of the imported file (and transitively, its own imported files)
before the current file.
A file can be imported multiple times, e.g., from several files in the same design.
Import cycles
(e.g., \verb|file1.ms| with \verb|import file2;| and \verb|file2.ms| with \verb|import file1;|).
are not allowed, and will cause a compiler error.

Finally, Minispec code may also import BSV code through a \textbf{bsvimport} statement, with syntax
\verb|bsvimport BsvFile;|. Following BSV conventions, \verb|BsvFile| must be uppercase.
This translates to an import statement in BSV, with its usual semantics.
